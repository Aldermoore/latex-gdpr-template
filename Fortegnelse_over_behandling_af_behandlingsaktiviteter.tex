% Dette dokument er en Latex implementering af GDPR-samtykke blanketten
% Aarhus Universitet tilbyder til projekter der inkluderer brugerundersøgelser el.lign. 

% Latex skal muligvis kompilerer TO gange for at f.eks. sidetal bliver opsat korrekt ! 

\documentclass[11pt, a4paper]{article}


%% Udfyld de nedenstående variabler med den relevante information. Latex vil selv sætte dem ind de rigtige steder, og tilpasse siden så det passer. 


\newcommand{\opgavetitel}{ Opgavetitel }	 		% Skriv navn på opgaven

\newcommand{\uddannelse}{ Bachelor i ... }	 		% Skriv uddannelse, f.eks. Bachelor/Kandidat i IT-Produktudvikling.  

\newcommand{\dataansvarligNavn}{ Dataansvarlig }	% Skriv navn på dataansvarlig 

\newcommand{\dataansvarligMail}{ name@example.org }	% Skriv mail, eller hold blank hvis der ikke kan kontaktes på mail. 
 
\newcommand{\dataansvarligTelefon}{ 12345678 }	 	% Skriv tlf nr, eller hold blank hvis der ikke kan kontaktes på tlf.  

\newcommand{\vejleder}{ Jens Hansen}				% Skriv navn på vejleder for kursus/projekt (hvis der er en)

\newcommand{\institution}{Aarhus Universitet} 		% Ændr institution til det passende

\newcommand{\dato}{ \today }						% Skriv dato for fortegnelse, eller behold som \today for dags dato


% Udfyld nedenstående med oplysninger til tabellen
\newcommand{\formaal}{Formålet med dataopsamlingen}	% Udfyld med formålet med dataopsamlingen

\newcommand{\personKategorier}{Børn og unge}		% Eksempelvis: Voksne i alderen 40 – 50 år eller børn med spiseforstyrrelser.

\newcommand{\personoplys}{Alder, køn, beskæftigelse}% Eksempelvis: vægt, højde, musiksmag, dialekt, testresultater

\newcommand{	\modtagere}{Vejleder og andre}			% Eksempelvis: Aarhus Universitet, studie-administration, underviser og vejleder, ekstern censor

\newcommand{\sletning}{1. august 2020}				% Angiv måned og år for udløb af klagefrist efter forventet tids-punkt for bedømmelse af opga-ven/projektet/specialet. Data kan opbevares indtil en eventuel klagesag er behandlet
\newcommand{\opbevaring}{Aarhus Universitet's Onedrive instans, der opfylder lovmæssige krav til opbevaring af personfølsomme oplysninger}% Eksempelvis: Aarhus Universitets Onedrive.

%%%%% Nedenfor skal der kun rettes i tabellen nederst %%%%%
%%%%%%%%%%%%%%%%%%%%%%%%%%%%%%%%%%%%%%%%%%%%%%%%%%%%%%%%%%%

% Imports
\usepackage{geometry}

\usepackage[utf8]{inputenc}
\usepackage[T1]{fontenc} 
\usepackage[danish]{babel} % Ændrer sproget til dansk, f.eks for indholdsfortegnelsen
\usepackage{microtype}  % Typografisk magi! Giver bl.a. pænere orddeling
\renewcommand{\danishhyphenmins}{22}  % Bedre dansk orddeling
\usepackage{lastpage}
\usepackage{fancyhdr}
\usepackage{graphicx}
\usepackage[dvipsnames]{xcolor}
\usepackage{libertine}            % Linux Libertine as text font

%Settings for the page layout
\setlength{\parindent}{0cm}
\geometry{
a4paper,
left = 2cm, 
right = 5.5cm,  
top = 1in, 
bottom = 3.6cm, 
}

% Header
\renewcommand{\headrulewidth}{0pt}
\setlength{\headheight}{48.14pt} 
\fancyhf{}
\fancyhead[RO]{\makebox[0pt][l]{\makebox[4cm][r]{
\textsf{
\begin{tabular}{l}
\textbf{\institution}      \\
\footnotesize{Dato: \dato}             \\
\vspace{2mm} \\
\footnotesize{Side \thepage/\pageref{LastPage}}\\
~~
\end{tabular}
}}}}






\begin{document}
\pagestyle{fancy}
~
\vspace{1cm}

\section*{Fortegnelse over behandling af behandlingsaktiviteter}~\\



Title på opgave/projekt/speciale:~~~\opgavetitel
\hrule


\vspace{5mm}

Uddannelse:~~~\uddannelse
\vspace{1mm}
\hrule

\vspace{5mm}


% Udfyld tabellen nedenfor med de relevante informationer
\begin{table}[h]

\begin{tabular}{ | l p{0.40\textwidth} | p{0.48\textwidth} | }
\hline
1. & Dataansvarlig:\newline Studerendes navn og kontaktoplysninger.  			& \dataansvarligNavn \newline \dataansvarligMail \newline \dataansvarligTelefon \\ \hline  % Hvis flere studerende har ad-gang til data foreligger fælles dataansvar. Udfyld erklæring for fælles ansvar. 
2. & Formålet med behandling af personoplysninger 								& \formaal     		\\ \hline
3. & Kategorier af personer, hvis data indgår i opgaven/projektet/specialet 	& \personKategorier	\\ \hline  % Eksempelvis: Voksne i alderen 40 – 50 år eller børn med spiseforstyrrelser. 
4. & Kategorier af personoplysninger 											& \personoplys  		\\ \hline  % Eksempelvis: vægt, højde, musiksmag, dialekt, testresultater
5. & Kategorier af modtagere af personoplysninger 								& \modtagere    		\\ \hline  % Eksempelvis: Aarhus Universitet, studie-administration, underviser og vejleder, ekstern censor
6. & Tidspunkt for sletning af personoplysningerne 								& \sletning 			\\ \hline % Angiv måned og år for udløb af klagefrist efter forventet tids-punkt for bedømmelse af opga-ven/projektet/specialet. Data kan opbevares indtil en eventuel klagesag er behandlet
7. & Til sikker opbevaring vil der blive anvendt følgende program (eller andet) 	& \opbevaring 		\\ \hline  % Eksempelvis: Aarhus Universitets Onedrive. 
\end{tabular}%

\end{table}



\end{document}
