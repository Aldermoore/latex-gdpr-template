% Dette dokument er en Latex implementering af GDPR-samtykke blanketten
% Aarhus Universitet tilbyder til projekter der inkluderer brugerundersøgelser el.lign. 

% Latex skal muligvis kompilerer TO gange for at f.eks. sidetal bliver opsat korrekt ! 

\documentclass[11pt, a4paper]{article}


%% Udfyld de nedenstående variabler med den relevante information. Latex vil selv sætte dem ind de rigtige steder, og tilpasse siden så det passer. 

\newcommand{\opgavetitel}{ Opgavetitel }	 		% Skriv navn på opgaven

\newcommand{\studerendeET}{ Katja Kaj }	 		% Skriv navn på studerende 1

\newcommand{\studerendeTO}{ Lise Lotte }	 		% Skriv navn på studerende 2

\newcommand{\studerendeTRE}{ Bente Bent }	 		% Skriv navn på studerende 3

\newcommand{\studerendeFIRE}{ Ole Olsen }	 	% Skriv navn på studerende 4

\newcommand{\institution}{Aarhus Universitet} 		% Ændr institution til det passende

\newcommand{\dato}{ \today }						% Skriv dato for underskrift, eller behold som \today for dags dato


\newcounter{number_of_students}
\setcounter{number_of_students}{4} % Skriv hvor mange studerende der er i gruppe. SKAL være 2, 3 eller 4

\newcounter{indhentelse}
\setcounter{indhentelse}{1} % Skriv nummeret på den studerende der har ansvaret for at indhente samtykkeeklæringer fra deltagerne

\newcounter{udlevering}
\setcounter{udlevering}{2} % Skriv nummeret på den studerende der har ansvaret for at udlevere kopi af samtykkeeklæringerne til deltagerne

\newcounter{fortegnelse}
\setcounter{fortegnelse}{1} % Skriv nummeret på den studerende der har ansvaret for at udarbejde fortegnelse af indsamlede personoplysninger fra deltagerne

\newcounter{opbevaring}
\setcounter{opbevaring}{2} % Skriv nummeret på den studerende der har ansvaret for at opbevare samtykkeeklæringer fra deltagerne fortroligt 

\newcounter{indberetning}
\setcounter{indberetning}{1} % Skriv nummeret på den studerende der har ansvaret for at indberette evt. sikkerhedsbrud til deltagere og datatilsynet

\newcounter{besvarelse}
\setcounter{besvarelse}{2} % Skriv nummeret på den studerende der har ansvaret for at besvare deltagernes senere spørgsmål ang. indsamlet data

\newcounter{sletning}
\setcounter{sletning}{1} % Skriv nummeret på den studerende der har ansvaret for slette personoplysningerne efter endt klagefrist. 



%%%%% Ret ikke i nedenstående %%%%%
%%%%%%%%%%%%%%%%%%%%%%%%%%%%%%%%%%%

% Imports
\usepackage{geometry}

\usepackage[utf8]{inputenc}
\usepackage[T1]{fontenc} 
\usepackage[danish]{babel} % Ændrer sproget til dansk, f.eks for indholdsfortegnelsen
\usepackage{microtype}  % Typografisk magi! Giver bl.a. pænere orddeling
\renewcommand{\danishhyphenmins}{22}  % Bedre dansk orddeling
\usepackage{lastpage}
\usepackage{fancyhdr}
\usepackage{graphicx}
\usepackage[dvipsnames]{xcolor}
\usepackage{libertine}            % Linux Libertine as text font

%Settings for the page layout
\setlength{\parindent}{0cm}
\geometry{
a4paper,
left = 2cm, 
right = 5.5cm,  
top = 1in, 
bottom = 3.6cm, 
}

% Header
\renewcommand{\headrulewidth}{0pt}
\setlength{\headheight}{48.14pt} 
\fancyhf{}
\fancyhead[RO]{\makebox[0pt][l]{\makebox[4cm][r]{
\textsf{
\begin{tabular}{l}
\textbf{\institution}      \\
\footnotesize{Dato: \dato}             \\
\vspace{2mm} \\
\footnotesize{Side \thepage/\pageref{LastPage}}\\
~~
\end{tabular}
}}}}


\newcommand{\two}{
\begin{table}[h]
\begin{tabular}{ | p{0.3\textwidth} | l | l | }
\hline
                                                                                                      & Studerende 1 & Studerende 2 \\ \hline
Indhentelse af samtykkeerklæringer fra deltagerne & \ifnum \value{indhentelse}=1 {x} \fi  & \ifnum \value{indhentelse}=2 {x} \fi  \\ \hline

Udlevere kopi af samtykkeerklæring til deltager & \ifnum \value{udlevering}=1 {x} \fi  & \ifnum \value{udlevering}=2 {x} \fi  \\ \hline

Udarbejdelse af fortegnelse over behandlingsaktiviteter & \ifnum \value{fortegnelse}=1 {x} \fi  & \ifnum \value{fortegnelse}=2 {x} \fi \\ \hline

Opbevaring af personoplysninger og herunder samtykkeerklæringer & \ifnum \value{opbevaring}=1 {x} \fi  & \ifnum \value{opbevaring}=2 {x} \fi \\ \hline

Indberetning af eventuelle sikkerhedsbrud til Datatilsynet og herefter orientering til databrud@au.dk & \ifnum \value{indberetning}=1 {x} \fi  & \ifnum \value{indberetning}=2 {x} \fi \\ \hline

Besvarelse af henvendelser fra deltagerne & \ifnum \value{besvarelse}=1 {x} \fi  & \ifnum \value{besvarelse}=2 {x} \fi  \\ \hline

Sletning af personoplysninger når opgaven/projektet/specialet er bedømt og klagefristen er udløbet & \ifnum \value{sletning}=1 {x} \fi  & \ifnum \value{sletning}=2 {x} \fi \\ \hline
\end{tabular}%
\end{table}
}
\newcommand{\three}{
\begin{table}[h]
\begin{tabular}{ | p{0.3\textwidth} | l | l | l |}
\hline	
                                                                                                     & Studerende 1 & Studerende 2 & Studerende 3 \\ \hline
Indhentelse af samtykkeerklæringer fra deltagerne & \ifnum \value{indhentelse}=1 {x} \fi  & \ifnum \value{indhentelse}=2 {x} \fi & \ifnum \value{indhentelse}=3 {x} \fi  \\ \hline

Udlevere kopi af samtykkeerklæring til deltager & \ifnum \value{udlevering}=1 {x} \fi  & \ifnum \value{udlevering}=2 {x} \fi & \ifnum \value{udlevering}=3 {x} \fi \\ \hline

Udarbejdelse af fortegnelse over behandlingsaktiviteter & \ifnum \value{fortegnelse}=1 {x} \fi  & \ifnum \value{fortegnelse}=2 {x} \fi & \ifnum \value{fortegnelse}=3 {x} \fi \\ \hline

Opbevaring af personoplysninger og herunder samtykkeerklæringer & \ifnum \value{opbevaring}=1 {x} \fi  & \ifnum \value{opbevaring}=2 {x} \fi & \ifnum \value{opbevaring}=3 {x} \fi \\ \hline

Indberetning af eventuelle sikkerhedsbrud til Datatilsynet og herefter orientering til databrud@au.dk & \ifnum \value{indberetning}=1 {x} \fi  & \ifnum \value{indberetning}=2 {x} \fi & \ifnum \value{indberetning}=3 {x} \fi \\ \hline

Besvarelse af henvendelser fra deltagerne & \ifnum \value{besvarelse}=1 {x} \fi  & \ifnum \value{besvarelse}=2 {x} \fi & \ifnum \value{besvarelse}=3 {x} \fi  \\ \hline

Sletning af personoplysninger når opgaven/projektet/specialet er bedømt og klagefristen er udløbet & \ifnum \value{sletning}=1 {x} \fi  & \ifnum \value{sletning}=2 {x} \fi & \ifnum \value{sletning}=3 {x} \fi \\ \hline
\end{tabular}%
\end{table}
}
\newcommand{\four}{
\begin{table}[h]
\begin{tabular}{ | p{0.3\textwidth} | l | l | l | l |}
\hline	
                                                                                                      & Studerende 1 & Studerende 2 & Studerende 3 & Studerende 4	\\ \hline
Indhentelse af samtykkeerklæringer fra deltagerne & \ifnum \value{indhentelse}=1 {x} \fi  & \ifnum \value{indhentelse}=2 {x} \fi & \ifnum \value{indhentelse}=3 {x} \fi & \ifnum \value{indhentelse}=4 {x} \fi \\ \hline

Udlevere kopi af samtykkeerklæring til deltager & \ifnum \value{udlevering}=1 {x} \fi  & \ifnum \value{udlevering}=2 {x} \fi & \ifnum \value{udlevering}=3 {x} \fi & \ifnum \value{udlevering}=4 {x} \fi \\ \hline

Udarbejdelse af fortegnelse over behandlingsaktiviteter & \ifnum \value{fortegnelse}=1 {x} \fi  & \ifnum \value{fortegnelse}=2 {x} \fi & \ifnum \value{fortegnelse}=3 {x} \fi & \ifnum \value{fortegnelse}=4 {x} \fi \\ \hline

Opbevaring af personoplysninger og herunder samtykkeerklæringer & \ifnum \value{opbevaring}=1 {x} \fi  & \ifnum \value{opbevaring}=2 {x} \fi & \ifnum \value{opbevaring}=3 {x} \fi & \ifnum \value{opbevaring}=4 {x} \fi \\ \hline

Indberetning af eventuelle sikkerhedsbrud til Datatilsynet og herefter orientering til databrud@au.dk & \ifnum \value{indberetning}=1 {x} \fi  & \ifnum \value{indberetning}=2 {x} \fi & \ifnum \value{indberetning}=3 {x} \fi & \ifnum \value{indberetning}=4 {x} \fi \\ \hline

Besvarelse af henvendelser fra deltagerne & \ifnum \value{besvarelse}=1 {x} \fi  & \ifnum \value{besvarelse}=2 {x} \fi & \ifnum \value{besvarelse}=3 {x} \fi & \ifnum \value{besvarelse}=4 {x} \fi \\ \hline

Sletning af personoplysninger når opgaven/projektet/specialet er bedømt og klagefristen er udløbet & \ifnum \value{sletning}=1 {x} \fi  & \ifnum \value{sletning}=2 {x} \fi & \ifnum \value{sletning}=3 {x} \fi & \ifnum \value{sletning}=4 {x} \fi \\ \hline
\end{tabular}%
\end{table}
}



\begin{document}
\pagestyle{fancy}
~
\vspace{1cm}

\textbf{Aftale om fælles dataansvar for personoplysninger indsamlet til brug for}~\\



\opgavetitel
\hrule
\vspace{2mm}
(Title på opgave/projekt/speciale)

\vspace{5mm}


Mellem\\ 

Navn på studerende 1: ~~~\studerendeET
\vspace{1mm}
\hrule

\vspace{5mm}

og 
\vspace{5mm}

Navn på studerende 2: ~~~\studerendeTO
\vspace{1mm}
\hrule

\vspace{5mm}

\ifnum \value{number_of_students}>2 {
og 
\vspace{5mm}

Navn på studerende 3: ~~~\studerendeTRE
\vspace{1mm}
\hrule

\vspace{5mm}
} \fi

\ifnum \value{number_of_students}>3 {
og 
\vspace{5mm}

Navn på studerende 4: ~~~\studerendeFIRE
\vspace{1mm}
\hrule

\vspace{5mm}
} \fi

% Hvis flere end to studerende – indsæt da disse ved at rette i skabelonen. 

\newpage
\subsection*{1. Fælles dataansvar}

1.1 Denne aftale fastsætter ansvarsfordelingen mellem Studerende 1 og Studerende 2 \ifnum \value{number_of_students}>2 {og Studerende 3 }\fi \ifnum \value{number_of_students}>3 { og Studerende 4 }\fi (De Studerende) i forbindelse behandling af persondata til brug for aflevering af ovennævnte opgave / projekt eller speciale som led i en uddannelse på \institution.\\

1.2 Der er mellem De Studerende enighed om, at der i forbindelse med opgaven / projektet / specialet foreligger et fælles dataansvar.

\subsection*{2. Overordnet opgavefordeling}

2.1 Parterne udfører følgende opgaver: 


\ifnum \value{number_of_students}=2% 
	{\two}%
\fi%

\ifnum \value{number_of_students}=3% 
	{\three}%
\fi%

\ifnum \value{number_of_students}=4%
	{\four}%
\fi%

\subsection*{3. Principper og behandlingshjemmel} 

3.1 Til brug for opgaven / projektet / specialet behandles personoplysninger indhentet fra personer, som har givet samtykke til behandlingen. Almindelige personoplysninger behandles efter bestemmelsen i persondataforordningens artikel 6, stk. 1, litra a) og følsomme personoplysninger behandles efter bestemmelsen i persondataforordningens artikel 9, stk. 2, litra a).\\

3.2 Studerende 1 og Studerende 2 er hver især ansvarlige for at overholde principperne for behandling af personoplysninger, i det omfang at reglerne finder anvendelse på den pågældendes ansvarsområder ifølge denne aftale.


\subsection*{4. De registreredes rettigheder}

4.1 Begge parter er ansvarlig for sikringen af de registreredes rettigheder gennem iagttagelse af nedenstående regler i databeskyttelsesforordningen:
\begin{itemize}
	\item den registreredes indsigtsret,
	\item ret til berigtigelse,
	\item ret til sletning (retten til at blive glemt), og
	\item ret til begrænsning af behandling.
\end{itemize}

Begge parter er ansvarlige for at udlevere en kopi af den underskrevne samtykkeblanket til de registrerede personer. Blanketten indeholder de oplysninger om databehandlingen, som er omhandlet i persondataforordningens artikel 13 om oplysningspligt ved indsamling af personoplysninger hos den registrerede.\\

Begge parter er ansvarlige for at underrette eventuelle modtagere af oplysningerne, hvis en oplysning er blevet berigtiget, begrænset eller slettet.\\

4.2 Parterne er ansvarlige for at bistå hinanden i det omfang, at dette er relevant og nødvendigt for, at begge parter kan efterleve forpligtelserne over for de registrerede.

\subsection*{5. Behandlingssikkerhed og dokumentation for overholdelse af databeskyttelsesforordningen}

5.1 Begge parter er ansvarlige for at overholde kravet i databeskyttelsesforordningens artikel 32 om behandlingssikkerhed.

\subsection*{6. Fortegnelse}

6.1 Der skal udarbejdes en fortegnelse.

\subsection*{7. Håndtering af brud på persondatasikkerheden}

7.1 Begge parter er ansvarlige for at straks at anmelde eventuelle brud på persondatasikkerheden til Datatilsynet. Se vejledning og kontaktoplysninger her:\\{\color{blue}https://www.datatilsynet.dk/anmeld-brud-paa-persondatasikkerheden/}

\subsection*{8. Klager}

8.1 Parterne er hver især ansvarlige for behandlingen af eventuelle klager fra registrere-de, hvis klagerne omhandler overtrædelse af bestemmelser i databeskyttelsesforordningen, for hvilke parten efter denne aftale er ansvarlig.

\subsection*{9. Orientering af den anden part}

9.1 Parterne orienterer hinanden om væsentlige forhold, der har betydning for den fælles behandling og denne aftale.

\subsection*{10. Ikrafttræden og ophør}

10.1 Denne aftale træder i kraft ved alle parters underskrift heraf.

10.2 Aftalen er gældende, så længe de omhandlede oplysninger behandles, eller indtil aftalen afløses af en ny aftale, som fastsætter ansvarsfordelingen i forbindelse med behandlingen.




\subsection*{Underskrift}

På vegne af studerende 1

Navn og underskrift~~~\studerendeET
\vspace{1mm}
\hrule
\vspace{5mm}
Stilling~~~
\vspace{1mm}
\hrule
\vspace{5mm}
Dato~~~\today
\vspace{1mm}
\hrule

\vspace{5mm}
På vegne af studerende 2

Navn og underskrift~~~\studerendeTO
\vspace{1mm}
\hrule
\vspace{5mm}
Stilling~~~
\vspace{1mm}
\hrule
\vspace{5mm}
Dato~~~\today
\vspace{1mm}
\hrule

\ifnum \value{number_of_students}>2 {
\vspace{5mm}
På vegne af studerende 3

Navn og underskrift~~~\studerendeTRE
\vspace{1mm}
\hrule
\vspace{5mm}
Stilling~~~
\vspace{1mm}
\hrule
\vspace{5mm}
Dato~~~\today
\vspace{1mm}
\hrule
} 
\fi

\ifnum \value{number_of_students}>3 {
\vspace{5mm}
På vegne af studerende 4

Navn og underskrift~~~\studerendeFIRE
\vspace{1mm}
\hrule
\vspace{5mm}
Stilling~~~
\vspace{1mm}
\hrule
\vspace{5mm}
Dato~~~\today
\vspace{1mm}
\hrule
}
\fi
\end{document}

