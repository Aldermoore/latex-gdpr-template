% Dette dokument er en Latex implementering af GDPR-samtykke blanketten
% Aarhus Universitet tilbyder til projekter der inkluderer brugerundersøgelser el.lign. 

% Latex skal muligvis kompilerer TO gange for at f.eks. sidetal bliver opsat korrekt ! 

\documentclass[11pt, a4paper]{article}


%% Udfyld de nedenstående variabler med den relevante information. Latex vil selv sætte dem ind de rigtige steder, og tilpasse siden så det passer. 


\newcommand{\opgavetitel}{ Opgavetitel }	 		% Skriv navn på opgaven

\newcommand{\studerendeET}{ studerende 1 }	 		% Skriv navn 

\newcommand{\studerendeTO}{ studerende 2 }	 		% Skriv navn 

\newcommand{\studerendeTRE}{ studerende 3 }	 		% Skriv navn 

\newcommand{\studerendeFIRE}{ studerende 4 }	 	% Skriv navn 

\newcommand{\institution}{Aarhus Universitet} % Ændr institution til det passende








%%%%% Nedenfor skal der kun rettes i tabellen over ansvar) %%%%%
%%%%%%%%%%%%%%%%%%%%%%%%%%%%%%%%%%%%%%%%%%%%%%%%%%%%%%

% Imports
\usepackage{geometry}

\usepackage[utf8]{inputenc}
\usepackage[T1]{fontenc} 
\usepackage[danish]{babel} % Ændrer sproget til dansk, f.eks for indholdsfortegnelsen
\usepackage{microtype}  % Typografisk magi! Giver bl.a. pænere orddeling
\renewcommand{\danishhyphenmins}{22}  % Bedre dansk orddeling
\usepackage{lastpage}
\usepackage{fancyhdr}
\usepackage{graphicx}
\usepackage[dvipsnames]{xcolor}
\usepackage{libertine}            % Linux Libertine as text font

%Settings for the page layout
\setlength{\parindent}{0cm}
\geometry{
a4paper,
left = 2cm, 
right = 5.5cm,  
top = 1in, 
bottom = 3.6cm, 
}

% Header
\renewcommand{\headrulewidth}{0pt}
\setlength{\headheight}{48.14pt} 
\fancyhf{}
\fancyhead[RO]{\makebox[0pt][l]{\makebox[4cm][r]{
\textsf{
\begin{tabular}{l}
\textbf{\institution}      \\
\footnotesize{Dato: \today}             \\
\vspace{2mm} \\
\footnotesize{Side \thepage/\pageref{LastPage}}\\
~~
\end{tabular}
}}}}






\begin{document}
\pagestyle{fancy}
~
\vspace{1cm}

\textbf{Aftale om fælles dataansvar for personoplysninger indsamlet til brug for}~\\



\opgavetitel
\hrule
\vspace{2mm}
(Title på opgave/projekt/speciale)

\vspace{5mm}


Mellem\\ 

Navn på studerende 1: ~~~\studerendeET
\vspace{1mm}
\hrule

\vspace{5mm}

og 
\vspace{5mm}

Navn på studerende 1: ~~~\studerendeTO
\vspace{1mm}
\hrule

\vspace{5mm}

og 
\vspace{5mm}

Navn på studerende 3: ~~~\studerendeTRE
\vspace{1mm}
\hrule

\vspace{5mm}

og 
\vspace{5mm}

Navn på studerende 3: ~~~\studerendeFIRE
\vspace{1mm}
\hrule

\vspace{5mm}


% Hvis flere end to studerende – indsæt da disse ved at rette i skabelonen. 

\newpage
\subsection*{1. Fælles dataansvar}

1.1 Denne aftale fastsætter ansvarsfordelingen mellem Studerende 1 og Studerende 2 i for- bindelse behandling af persondata til brug for aflevering af ovennævnte opgave / projekt eller speciale som led i en uddannelse på Aarhus Universitet.\\

1.2 Der er mellem Studerende 1 og Studerende 2 enighed om, at der i forbindelse med op- gaven/projektet/specialet foreligger et fælles dataansvar.

\subsection*{2. Overordnet opgavefordeling}

2.1 Parterne udfører følgende opgaver: 


\begin{table}[]
\begin{tabular}{ | p{0.3\textwidth} | l | l | l |}
\hline	
                                                                                                      & studerende 1 & studerende 2 & Studerende 3 & Studerende 4 	\\ \hline
Indhentelse af samtykkeerklæringer fra deltagerne                                                     &       x       &              & 			   & 				\\ \hline
Udlevere kopi af samtykkeerklæring til deltager                                                       &       x       &              & 			   & 				\\ \hline
Udarbejdelse af fortegnelse over behandlingsaktiviteter                                               &       x       &              & 				& 				\\ \hline
Opbevaring af personoplysninger og herunder samtykkeerklæringer                                       &       x       &              & 				& 				\\ \hline
Indberetning af eventuelle sikkerhedsbrud til Datatilsynet og herefter orientering til databrud@au.dk &       x       &              & 				& 				\\ \hline
Besvarelse af henvendelser fra deltagerne                                                             &       x       &              & 				& 				\\ \hline
Sletning af personoplysninger når opgaven/projektet/specialet er bedømt og klagefristen er udløbet    &       x       &              & 	           & 				\\ \hline
\end{tabular}%
\end{table}

\subsection*{3. Principper og behandlingshjemmel} 

3.1 Til brug for opgaven / projektet / specialet behandles personoplysninger indhentet fra personer, som har givet samtykke til behandlingen. Almindelige personoplysninger behandles efter bestemmelsen i persondataforordningens artikel 6, stk. 1, litra a) og følsomme personoplysninger behandles efter bestemmelsen i persondataforordningens artikel 9, stk. 2, litra a).\\

3.2 Studerende 1 og Studerende 2 er hver især ansvarlige for at overholde principperne for behandling af personoplysninger, i det omfang at reglerne finder anvendelse på den pågældendes ansvarsområder ifølge denne aftale.


\subsection*{4. De registreredes rettigheder}

4.1 Begge parter er ansvarlig for sikringen af de registreredes rettigheder gennem iagttagelse af nedenstående regler i databeskyttelsesforordningen:
\begin{itemize}
	\item den registreredes indsigtsret,
	\item ret til berigtigelse,
	\item ret til sletning (retten til at blive glemt), og
	\item ret til begrænsning af behandling.
\end{itemize}

Begge parter er ansvarlige for at udlevere en kopi af den underskrevne samtykkeblanket til de registrerede personer. Blanketten indeholder de oplysninger om databehandlingen, som er omhandlet i persondataforordningens artikel 13 om oplysningspligt ved indsamling af personoplysninger hos den registrerede.\\

Begge parter er ansvarlige for at underrette eventuelle modtagere af oplysningerne, hvis en oplysning er blevet berigtiget, begrænset eller slettet.\\

4.2 Parterne er ansvarlige for at bistå hinanden i det omfang, at dette er relevant og nødvendigt for, at begge parter kan efterleve forpligtelserne over for de registrerede.

\subsection*{5. Behandlingssikkerhed og dokumentation for overholdelse af databeskyttelsesforordningen}

5.1 Begge parter er ansvarlige for at overholde kravet i databeskyttelsesforordningens artikel 32 om behandlingssikkerhed.

\subsection*{6. Fortegnelse}

6.1 Der skal udarbejdes en fortegnelse.

\subsection*{7. Håndtering af brud på persondatasikkerheden}

7.1 Begge parter er ansvarlige for at straks at anmelde eventuelle brud på persondatasikkerheden til Datatilsynet. Se vejledning og kontaktoplysninger her:\\{\color{blue}https://www.datatilsynet.dk/anmeld-brud-paa-persondatasikkerheden/}

\subsection*{8. Klager}

8.1 Parterne er hver især ansvarlige for behandlingen af eventuelle klager fra registrere-de, hvis klagerne omhandler overtrædelse af bestemmelser i databeskyttelsesforordningen, for hvilke parten efter denne aftale er ansvarlig.

\subsection*{9. Orientering af den anden part}

9.1 Parterne orienterer hinanden om væsentlige forhold, der har betydning for den fælles behandling og denne aftale.

\subsection*{10. Ikrafttræden og ophør}

10.1 Denne aftale træder i kraft ved begge parters underskrift heraf.

10.2 Aftalen er gældende, så længe de omhandlede oplysninger behandles, eller indtil aftalen afløses af en ny aftale, som fastsætter ansvarsfordelingen i forbindelse med be-handlingen.




\subsection*{Underskrift}

På vegne af \studerendeET

Navn og underskrift~~~\studerendeET
\vspace{1mm}
\hrule
\vspace{5mm}
Stilling~~~
\vspace{1mm}
\hrule
\vspace{5mm}
Dato~~~\today
\vspace{1mm}
\hrule

\vspace{5mm}
På vegne af \studerendeTO

Navn og underskrift~~~\studerendeTO
\vspace{1mm}
\hrule
\vspace{5mm}
Stilling~~~
\vspace{1mm}
\hrule
\vspace{5mm}
Dato~~~\today
\vspace{1mm}
\hrule


\vspace{5mm}
På vegne af \studerendeTRE

Navn og underskrift~~~\studerendeTRE
\vspace{1mm}
\hrule
\vspace{5mm}
Stilling~~~
\vspace{1mm}
\hrule
\vspace{5mm}
Dato~~~\today
\vspace{1mm}
\hrule


\vspace{5mm}
På vegne af \studerendeFIRE

Navn og underskrift~~~\studerendeFIRE
\vspace{1mm}
\hrule
\vspace{5mm}
Stilling~~~
\vspace{1mm}
\hrule
\vspace{5mm}
Dato~~~\today
\vspace{1mm}
\hrule

\end{document}
