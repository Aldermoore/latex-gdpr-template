% Dette dokument er en Latex implementering af GDPR-samtykke blanketten
% Aarhus Universitet tilbyder til projekter der inkluderer brugerundersøgelser el.lign. 

% Latex skal muligvis kompilerer TO gange for at f.eks. sidetal bliver opsat korrekt ! 

\documentclass[11pt, a4paper]{article}


%% Udfyld de nedenstående variabler med den relevante information. Latex vil selv sætte dem ind de rigtige steder, og tilpasse siden så det passer. 

\newcommand{\vejleder}{ Navn på vejleder }			% Skriv navn på vejleder for kursus/projekt
\newcommand{\dataansvarlig}{ Navn på dataansvarlig }	% Skriv navn på den dataansvarlige studerende 
\newcommand{\mail}{ kontakt@dataansvarlig } 			% Skriv email på dataansvarlig eller person der har kontakt-rollen
\newcommand{\opgavetitel}{ Opgavetitel }				% Skriv navn på opgaven
\newcommand{\opgavebeskrivelse}{

Beskrivelse af projektet her...

}

\newcommand{\institution}{Aarhus Universitet} % Ændr institution til det passende









%%%%% ÆNDR IKKE I DET NEDENSTÅENDE %%%%%
%%%%%%%%%%%%%%%%%%%%%%%%%%%%%%%%%%%%%%%%%%%%%%%%%%%%%%

% Imports
\usepackage{geometry}

\usepackage[utf8]{inputenc}
\usepackage[T1]{fontenc} 
\usepackage[danish]{babel} % Ændrer sproget til dansk, f.eks for indholdsfortegnelsen
\usepackage{microtype}  % Typografisk magi! Giver bl.a. pænere orddeling
\renewcommand{\danishhyphenmins}{22}  % Bedre dansk orddeling
\usepackage{lastpage}
\usepackage{fancyhdr}
\usepackage{graphicx}
\usepackage[dvipsnames]{xcolor}
\usepackage{libertine}            % Linux Libertine as text font

%Settings for the page layout
\setlength{\parindent}{0cm}
\geometry{
a4paper,
left = 2cm, 
right = 5.5cm,  
top = 1in, 
bottom = 3.6cm, 
}

% Header
\renewcommand{\headrulewidth}{0pt}
\setlength{\headheight}{48.14pt} 
\fancyhf{}
\fancyhead[RO]{\makebox[0pt][l]{\makebox[4cm][r]{
\textsf{
\begin{tabular}{l}
\textbf{\institution}      \\
\footnotesize{Dato: \today}             \\
\vspace{2mm} \\
\footnotesize{Side \thepage/\pageref{LastPage}}\\
~~
\end{tabular}
}}}}






\begin{document}
\pagestyle{fancy}
~
\vspace{1cm}

\section*{Samtykke til behandling af personoplysninger}

Dataansvarlig studerende: ~~~\dataansvarlig
\hrule

\vspace{5mm}

Titel på opgave/projekt/speciale: ~~~\opgavetitel
\hrule

\vspace{5mm}

Vejleder på opgave/projekt/speciale: ~~~\vejleder
\hrule

\vspace{5mm}

Beskrivelse af projektet, herunder formålet med databehandlingen og hvilke personoplysninger der behandles: \\

\textit{
\opgavebeskrivelse
}

~\\

\hrule 
\vspace{6mm}

Jeg giver hermed samtykke til at ovennævnte studerende må behandle oplysninger om mig i forbindelse med sin uddannelse på Aarhus Universitet. Mine personoplysninger vil indgå i ovennævnte opgave/projekt/speciale. Jeg giver samtykke til, at:
\begin{itemize}
\item mine oplysninger må behandles i opgaven/projektet/specialet
\item mine oplysninger må videregives til en eller flere studerende, som skriver opgaven/projektet eller specialet i fællesskab. De studerende har fælles dataansvar
\item mine oplysninger må videregives til Aarhus Universitet og til en eventuel ekstern censor i forbindelse med vejledning og bedømmelse
\item mine oplysninger må offentliggøres i anonymiseret form i forbindelse med offentliggørelse af projektet eller specialet. Dette punkt kan helt fjernes hvis det vurderes ikke at være nødvendigt. Vurderingen skal ske ved hver enkelt projekt.
\end{itemize}

\vspace{2mm}

Dato: 
\vspace{5mm}

Navn: 
\vspace{5mm}

Underskrift: 
\vspace{5mm}

Samtykket kan til enhver tid trækkes tilbage med virkning for fremtiden. Dette sker via henvendelse til denne mail:  {\color{blue}\mail}


\paragraph{Information til den registrerede}~\\
Efter reglerne i persondataforordningen skal den studerende som dataansvarlig informere de registrerede personer om deres rettigheder i forbindelse med behandlingen af oplysningerne. Den studerende registrerer og behandler personoplysninger med hjemmel persondataforordningens artikel 6, stk. 1, litra a). Følsomme data, dvs. helbredsdata eller data om race eller etnisk oprindelse, politisk, religiøs eller filosofisk overbevisning eller fagforeningsmæssigt tilhørsforhold registreres og behandles med hjemmel i persondataforordningens artikel 9, stk. 2, litra a). Begge regler giver adgang til at behandle oplysninger, når den registrerede har givet udtrykkeligt samtykke.

\paragraph{Behandling og opbevaring}~\\
Den studerende behandler personoplysningerne fortroligt. Oplysningerne vil blive opbevaret indtil opgaven/projektet/specialet er bedømt og klagefristen i forbindelse med bedømmelsen er udløbet.

\paragraph{Videregivelse af oplysninger}~\\
Oplysningerne vil ikke blive videregivet til andre medmindre der er givet samtykke her-til.

\paragraph{Dataindsigt}~\\
Registrerede personer kan når som helst rette henvendelse til den studerende med henblik på at få kopi af oplysningerne.

\paragraph{Berigtigelse af oplysninger}~\\
Hvis den registrerede person mener, at der er registreret forkerte oplysninger, kan man bede den studerende om at berigtige oplysningerne. Det vil sige, at den studerende retter oplysningerne eller noterer, at oplysningerne er forkerte og registrerer de rigtige oplysninger. Den registrerede person har krav på, at den studerende ser bort fra oplysningerne indtil det er afgjort, hvilke oplysninger, der er rigtige.

\paragraph{Tilbagekaldelse af samtykke og sletning af oplysninger}~\\
Hvis den studerende har indhentet et samtykke fra den registrerede person til at be-handle oplysningerne, vil den registrerede til enhver tid kunne tilbagekalde samtykket. 

Den studerende kan derfor ikke fortsætte med at behandle oplysningerne efter samtykket er trukket tilbage.
Den registrerede har ret til at få slettet oplysninger, som den studerende har registreret om den pågældende, hvis oplysningerne ikke længere er nødvendige til det formål de blev indsamlet til. Oplysningerne skal også slettes, hvis den registrerede tilbagekalder samtykket til behandlingen eller hvis oplysningerne ved en fejl er blevet behandlet ulovligt. Den registrerede har ikke krav på sletning af oplysninger, som er arkiverede efter arkivlovens regler i universitetets arkivsystem.

\paragraph{Klage til Datatilsynet}~\\
Registrerede personer kan klage over behandlingen af oplysningerne til Datatilsynet: {\color{blue}dt@datatilsynet.dk}.

\end{document}
